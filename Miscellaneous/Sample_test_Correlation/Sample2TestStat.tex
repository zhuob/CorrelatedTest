
\documentclass[12pt,oneside]{book}
%%%%%%%%%%%%%%%%%%%%%%%%%%%%%%%%%%%%%%%%%%%%%%%%%%%%%%%%%%%%%%%%%%%%%%%%%%%%%%%%%%%%%%%%%%%%%%%%%%%%%%%%%%%%%%%%%%%%%%%%%%%%%%%%%%%%%%%%%%%%%%%%%%%%%%%%%%%%%%%%%%%%%%%%%%%%%%%%%%%%%%%%%%%%%%%%%%%%%%%%%%%%%%%%%%%%%%%%%%%%%%%%%%%%%%%%%%%%%%%%%%%%%%%%%%%%
\usepackage{fancyhdr}
\usepackage{setspace}
\usepackage{amsmath}
\usepackage{amssymb}
\usepackage{bm}
\usepackage{graphicx}
\usepackage[mathcal,mathscr]{euscript}
\usepackage[usenames,dvipsnames]{color}
\usepackage[rflt]{floatflt}
\usepackage{latexsym, enumerate}
\usepackage{graphicx, amsmath,latexsym,times,color}
\usepackage{amssymb}
\usepackage{amsbsy}
\usepackage{epsfig}
\usepackage{color}
\usepackage{enumerate,verbatim}
\usepackage{gensymb}
\usepackage{multirow}
\usepackage{enumerate}
\usepackage{booktabs}
\usepackage[singlelinecheck=false, aboveskip=0pt]{caption}
\usepackage{subcaption}
\usepackage{hyperref}
\usepackage{geometry}
\usepackage[toc,page]{appendix}
\usepackage[authoryear,round]{natbib}
\bibliographystyle{apalike}



\hypersetup{
    colorlinks,
    citecolor=blue,
    filecolor=blue,
    linkcolor=blue,
    urlcolor=blue
}

\setcounter{MaxMatrixCols}{10}
%TCIDATA{OutputFilter=LATEX.DLL}
%TCIDATA{Version=5.50.0.2953}
%TCIDATA{<META NAME="SaveForMode" CONTENT="1">}
%TCIDATA{BibliographyScheme=Manual}
%TCIDATA{LastRevised=Tuesday, December 02, 2014 07:50:12}
%TCIDATA{<META NAME="GraphicsSave" CONTENT="32">}

%\input{tcilatex}
\newcommand{\E}{{\rm I\kern-.3em E}}
\newcommand{\Var}{\mathrm{Var}}
\newcommand{\prb}{{\rm I\kern-.3em P}}
%\newtheorem{theorem}{Theorem}[section]
%\newtheorem{lemma}[theorem]{Lemma}
%\newtheorem{proposition}[theorem]{Proposition}
%\newtheorem{corollary}[theorem]{Corollary}
\newcommand{\xl}[1]{\textcolor{blue}{#1}}
%\input{tcilatex}

%%%%%%%%%%%%%%%%%%%%%%%%%%%%%%%%%%%%%%%%%
% University Assignment Title Page 
% LaTeX Template
% Version 1.0 (27/12/12)
%
% This template has been downloaded from:
% http://www.LaTeXTemplates.com
%
% Original author:
% WikiBooks (http://en.wikibooks.org/wiki/LaTeX/Title_Creation)
%
% License:
% CC BY-NC-SA 3.0 (http://creativecommons.org/licenses/by-nc-sa/3.0/)
% 
% Instructions for using this template:
% This title page is capable of being compiled as is. This is not useful for 
% including it in another document. To do this, you have two options: 
%
% 1) Copy/paste everything between \begin{document} and \end{document} 
% starting at \begin{titlepage} and paste this into another LaTeX file where you 
% want your title page.
% OR
% 2) Remove everything outside the \begin{titlepage} and \end{titlepage} and 
% move this file to the same directory as the LaTeX file you wish to add it to. 
% Then add \input{./title_page_1.tex} to your LaTeX file where you want your
% title page.
%
%%%%%%%%%%%%%%%%%%%%%%%%%%%%%%%%%%%%%%%%%

%----------------------------------------------------------------------------------------
%	PACKAGES AND OTHER DOCUMENT CONFIGURATIONS
%----------------------------------------------------------------------------------------

\begin{document}

\newpage

\newgeometry{top=1in,bottom=1in,right=1in,left=1in}
\pagenumbering{arabic}
\pagestyle{fancy}
%\chapter{Introduction} % Main chapter title

%\label{Chapter1} % Change X to a consecutive number; for referencing this chapter elsewhere, use \ref{ChapterX}
\section*{Relation of $Z$-test statistics correlation and sample correlation}

%\lhead{Chapter 1. \emph{Introduction}} % Change X to a consecutive number; this is for the header on each page - perhaps a shortened title
\cfoot{}
\rhead{\thepage}

Under normal assumption. Let's assume gene 1 has expression level $G_1 = (X_{11}, \ldots, X_{1m}, Y_{11}, \ldots, Y_{1n})$ containing two treatments, similar for gene 2 $G_2 = (X_{21}, \ldots, X_{2m}, Y_{21}, \ldots, Y_{2n})$. Suppose in a general sense
\begin{equation}\label{assumption1}
  X_{i1}\sim N(\mu_1, \sigma^2_1), ~~ Y_{j1}\sim N(\mu_2, \sigma^2_2), ~~~X_{2i}\sim N(\mu_3, \sigma^2_3), ~~ Y_{2j}\sim N(\mu_4, \sigma^2_4)
\end{equation}
\subsection*{Gene-Gene correlation}
The correlation between two genes is defined, in my simulation, as 
\begin{align}\label{correlation}
\rho & = Cor(X_{1i}, X_{2i}) =  Cor(Y_{1j}, Y_{2j}) \\ \notag
\end{align}


\subsection*{Sample correlation}
Let  $\bar{G}_1 = (\bar{X}_1, \bar{Y}_1)$ and $\bar{G}_2 = (\bar{X}_2, \bar{Y}_2)$, the sample correlation is defined as  
\begin{align} \label{SampleCorrelation}
Cor(G_1, G_2) &= \frac{Cov(G_1, G_2)}{\sqrt{Var(G_1)Var(G_2)}} \\ 
&= \frac{\sum_i(X_{1i} - \bar{X}_1)(X_{2i} -\bar{X}_2) + \sum_j(Y_{1j}-\bar{Y}_1)(Y_{2j}-\bar{Y}_2)}{\sqrt{\sum_i(X_{i1}-\bar{X}_1)^2+ \sum_j(Y_{1j}-\bar{Y}_1)^2}\sqrt{\sum_i(X_{2i}-\bar{X}_2)^2 + \sum_j(Y_{2j}-\bar{Y}_2)^2}} \notag
\end{align}

\subsection*{Z test statistics}
Now the test statistics are 
\[Z_1 = \frac{\bar{X}_{1}-\bar{Y}_{1} }{\sqrt{\frac{\sigma^2_1}{m} + \frac{\sigma^2_2}{n}}}\sim N(\mu_1-\mu_2, 1)\]
\[Z_2 = \frac{\bar{X}_2-\bar{Y}_2}{\sqrt{\frac{\sigma^2_3}{m} + \frac{\sigma^2_4}{n}}} \sim N(\mu_3-\mu_4, 1)\]
therefore assuming samples are independent of each other
\begin{equation}\notag
\begin{aligned}
Cov(Z_1, Z_2 ) &= \frac{1}{c_0}Cov(\bar{X}_1-\bar{Y}_1, \bar{X}_2-\bar{Y}_2)\\
 & = \frac{1}{c_0}[Cov(\bar{X}_1, \bar{X}_2) + Cov(\bar{Y}_1, \bar{Y}_2)]\\
 & = \frac{1}{c_0}[Cov(\sum_{i=1}^m X_{1i} , \sum_{i=1}^m X_{2i})/m^2 + Cov(\sum_{j=1}^n Y_{1j}, \sum_{j=1}^n Y_{2j})/n^2]\\
 & =  \frac{1}{c_0}[\sum_{i=1}^mCov( X_{1i} ,  X_{2i})/m^2 + \sum_{j=1}^n Cov( Y_{1j},  Y_{2j})/n^2] \\
 & = \frac{1}{c_0}[\frac{\rho\sigma_1\sigma_3 }{m} + \frac{\rho \sigma_2 \sigma_4}{n}]
 \end{aligned}
\end{equation}
where $c_0= \left(\frac{\sigma^2_1}{m} + \frac{\sigma^2_2}{n}\right)\left(\frac{\sigma^2_3}{m} + \frac{\sigma^4_2}{n}\right)$. Note that $Var(Z_1)= 1$
we have 
\begin{align} \label{Z-correlation}
\rho(Z_1, Z_2) = \frac{Cov(Z_1, Z_2)}{\sqrt{Var{(Z_1)}}\sqrt{Var{(Z_2)}}} = \rho \cdot \frac{\frac{\sigma_1\sigma_3 }{m} + \frac{\sigma_2 \sigma_4}{n}}{\sqrt{(\frac{\sigma^2_1}{m} + \frac{\sigma^2_2}{n})(\frac{\sigma^2_3}{m} + \frac{\sigma^2_4}{n})}}
\end{align}

This equals to $\rho $ if and only if
\begin{align}\label{condition}
\frac{\frac{\sigma_1\sigma_3 }{m} + \frac{\sigma_2 \sigma_4}{n}}{\sqrt{(\frac{\sigma^2_1}{m} + \frac{\sigma^2_2}{n})(\frac{\sigma^2_3}{m} + \frac{\sigma^2_4}{n})}} = 1 \Rightarrow \sigma_1\sigma_4= \sigma_2\sigma_3 
\end{align}

 In a typical gene expression analysis, it is assumed that for the same gene, variance across different treatments are constant. Therefore $\sigma_1=\sigma_2$ and $\sigma_3= \sigma_4$, and the gene correlation and test statistic correlation are the same for $Z$-test.

\section*{Relation of $T$-test statistics correlation and sample correlation}
Similarly, under assumption (\ref{assumption1}), but we don't know the $\sigma_1^2, \sigma_2^2,\sigma_3^2, \sigma_4^2$. For now, we assume within the same gene, the expression levels have the same variance (i.e., $\sigma_1^2=\sigma_2^2, \sigma_3^2,=\sigma_4^2,$). 

\subsection*{Pooled variance}
The $T$ test statistics (pooled variance) for gene 1 and gene 2 are 
\begin{align}\label{t-test}
 T_1 = \frac{\bar{X}_1 - \bar{Y}_1}{S_1\sqrt{\frac{1}{m}+\frac{1}{n}}},  T_2 = \frac{\bar{X}_2-\bar{Y}_2}{S_2\sqrt{\frac{1}{m}+\frac{1}{n}}}
\end{align}
where 
\[S_1^2 =\frac{(m-1)S_{X_1}^2+ (n-1)S_{Y_1}^2}{m+n-2}, ~~~ S_2^2 =\frac{(m-1)S_{X_2}^2+ (n-1)S_{Y_2}^2}{m+n-2}\]

Since $S_{X_1}^2, S_{Y_1}^2, S_{X_2}^2, S_{Y_2}^2$ are consistent estimators of $\sigma_1^2, \sigma_2^2,\sigma_3^2, \sigma_4^2$ respectively. Therefore 
\[    S_1^2  \stackrel {d}{ \rightarrow} \frac{(m-1)\sigma_1^2 +(n-1)\sigma_2^2}{m + n -2}~~ \stackrel{def}{=}\delta_1^2,  ~~~~~~  S_2^2  \stackrel {d}{ \rightarrow} \frac{(m-1)\sigma_3^2 +(n-1)\sigma_4^2}{m + n -2}~~ \stackrel{def}{=}\delta_2^2, \]
Therefore we have   
\begin{align*}
Cov(T_1, T_2) & \approx \frac{1}{\delta_1\delta_2 (\frac{1}{m} + \frac{1}{n})}Cov(\bar{X}_1 - \bar{Y}_1, \bar{X}_2 - \bar{Y}_2) \\ \notag
& = \frac{1}{\delta_1\delta_2 (\frac{1}{m} + \frac{1}{n})} \left[\frac{\rho\sigma_1\sigma_3 }{m} + \frac{\rho \sigma_2 \sigma_4}{n}\right] \\ \notag
\end{align*}
Note that 
\[Var(T_1) \approx \frac{1}{\delta_1^2(\frac{1}{m} + \frac{1}{n})}Var(\bar{X}_1 -\bar{Y}_1) = \frac{1}{\delta_1^2(\frac{1}{m} + \frac{1}{n})} (\frac{\sigma_1^2}{m} + \frac{\sigma_2^2}{n}) \]
\[Var(T_2) \approx \frac{1}{\delta_1^2(\frac{1}{m} + \frac{1}{n})}Var(\bar{X}_2 -\bar{Y}_2) = \frac{1}{\delta_2^2(\frac{1}{m} + \frac{1}{n})} (\frac{\sigma_3^2}{m} + \frac{\sigma_4^2}{n}) \]
Therefore 
\begin{align}\label{T-correlation}
\rho(T_1, T_2)  = \frac{Cov(T_1, T_2)}{\sqrt{Var(T_1)Var(T_2)}} \approx \rho \cdot \frac{\frac{\sigma_1\sigma_3 }{m} + \frac{\sigma_2 \sigma_4}{n}}{\sqrt{(\frac{\sigma^2_1}{m} + \frac{\sigma^2_2}{n})(\frac{\sigma^2_3}{m} + \frac{\sigma^2_4}{n})}}
\end{align}
It resembles (\ref{Z-correlation}). 

\subsection*{Unequal variances}
The $T$ test statistics in this case are 
\begin{align*}\label{ttest-unequal}
 T_1 = \frac{\bar{X}_1 - \bar{Y}_1}{S_1},  T_2 = \frac{\bar{X}_2-\bar{Y}_2}{S_2}
\end{align*}
where 
\[S_1^2 = \frac{S_{X_1}^2}{m} + \frac{S_{Y_1}^2}{n}, ~~ S_2^2 = \frac{S_{X_2}^2}{m} + \frac{S_{Y_2}^2}{n}\]
and we have 
\[    S_1^2  \stackrel {d}{ \rightarrow} \frac{\sigma_1^2}{m} + \frac{\sigma_2^2}{n}~~ \stackrel{def}{=}\delta_1^2,  ~~~~~~  S_2^2  \stackrel {d}{ \rightarrow} \frac{\sigma_3^2}{m} + \frac{\sigma_4^2}{n}~~ \stackrel{def}{=}\delta_2^2, \]
The correlation of $T_1$ and $T_2$ can be calculated by 
\[Cov(T_1, T_2) \approx \frac{1}{\delta_1\delta_2}\left[\frac{\rho\sigma_1\sigma_3 }{m} + \frac{\rho \sigma_2 \sigma_4}{n}\right]\]
\[Var(T_1)  \approx\frac{1}{\delta_1^2}(\frac{\sigma_1^2}{m} + \frac{\sigma_2^2}{n}) , ~~~Var(T_2)  \approx\frac{1}{\delta_2^2}(\frac{\sigma_3^2}{m} + \frac{\sigma_4^2}{n}) \]
expression (\ref{T-correlation}) holds for unequal variance case, too. 
\section*{Sample correlation and Gene correlation}
We have established the equality of inter-gene correlation and test statistic correlation (for both $Z$ and $T$ tests). However, it is interesting to point out that sample correlation does not mean inter-gene correlation. \\

An estimate of inter-gene correlation
\[\hat{\rho} = \frac{\sum_{i=1}^m (x_{i1}-\bar{x}_1)(x_{i2} -\bar{x}_2)}{\sqrt{\sum_{i=1}^m(x_{i1}-\bar{x}_1)^2}\sqrt{\sum_{i=1}^m(x_{i2}-\bar{x}_2)^2}}\]
or 
\[\hat{\rho} = \frac{\sum_{j=1}^n(y_{i1}-\bar{y}_1)(y_{i2} -\bar{y}_2)}{\sqrt{\sum_{j=1}^n(y_{i1}-\bar{y}_1)^2}\sqrt{\sum_{j=1}^n(y_{i2}-\bar{y}_2)^2}}\]

However, if we define the sample correlation as (\ref{SampleCorrelation}),  that is, 
\begin{align} \notag
Cor(G_1, G_2) &= \frac{Cov(G_1, G_2)}{\sqrt{Var(G_1)Var(G_2)}} \\ 
&= \frac{\sum_i(X_{1i} - \bar{X}_1)(X_{2i} -\bar{X}_2) + \sum_j(Y_{1j}-\bar{Y}_1)(Y_{2j}-\bar{Y}_2)}{\sqrt{\sum_i(X_{i1}-\bar{X}_1)^2+ \sum_j(Y_{1j}-\bar{Y}_1)^2}\sqrt{\sum_i(X_{2i}-\bar{X}_2)^2 + \sum_j(Y_{2j}-\bar{Y}_2)^2}} \notag
\end{align}

\textbf{NOTE}:  as long as (\ref{condition}) holds
\[  Cor(Z_1, Z_2 )\approx Cor(G_1,G_2) \]
where $\bar{G}_1 = (\bar{X}_1, \bar{Y}_1)$ and $\bar{G}_2 = (\bar{X}_2, \bar{Y}_2)$. \\

%If (\ref{condition}) is violated, the correlation  would drastically change. For example,  if $\sigma_1/\sigma_2$ and $\sigma_3/\sigma_4$ are disproportional in different directions. For example, if $\sigma_1/\sigma_2 = 0.1$ and $\sigma_3/\sigma_4=10$,  $Cov(X_{1i}, X_{2i}) = Cov(Y_{1j}, Y_{2j})= \rho$ generally holds, but $Cov(G_1, G_2)$ might be very different from $\rho$. 

\subsection*{DE or NOT DE matters}
If we look at the $z$-test statistic and $t$-test statistic, 
\begin{align*}
Z = \frac{\bar{X}_{1}-\bar{Y}_{1} }{\sqrt{\frac{\sigma^2_1}{m} + \frac{\sigma^2_2}{n}}},~~~T = \frac{\bar{X}_1 -\bar{Y}_1}{\sqrt{\frac{S_{X_1}^2}{m} + \frac{S_{Y_1}^2}{n}}}
\end{align*}
Although we already established that $S_{X_1}^2 \stackrel{d}{\rightarrow} \sigma_1^2$ and $S_{Y_1}^2 \stackrel{d}{\rightarrow} \sigma_2^2$,  the correlation of $T$ statistics will not remain the same as correlation of $Z$ statistics.  Two reasons for that:
\begin{enumerate}
\item if $m$ and $n$ are small, then $S_{X_1}^2$ and $S_{Y_1}^2$ cannot be accurately estimated
\item if $m$ and $n$ are large, then the denominator will be very small, in which case a slight difference between the sample variance  $S_{X_1}^2$ and true variance  $\sigma^2_1$ will augment the difference of test statistics. 
\end{enumerate}
\subsection*{General Conclusion}
The way we define the sample correlations really matters! \\

If we define the sample correlation as $\rho = \frac{1}{2}(\rho_1 + \rho_2)$ where $\rho_1 = Cor(X_1, X_2), \rho_2=Cor(Y_1, Y_2)$, then sample correlation equals to test statistics correlation as long as (\ref{condition}) holds. We do care about whether there is DE or not (simulation study shows that). \\

For $Z$ test, the correlation between $Z$ statistics and correlation between expression value generally match, since within a gene, we assume they have the same variance across two treatments (i.e., $\sigma_1^2=\sigma_2^2, \sigma_3^2,=\sigma_4^2,$). \\

However, for a typical $T$-test, the mean difference between the comparison will play a role. Denote 
$Y_{1i}'= Y_{1i}-\bar{Y}_1$ and similar for $X_1, X_2$ and $Y_2$.  Unless the difference between $Y'_i$ and $Y_i$ is negligible (No DE), the sample correlation we obtained is different from T correlation. But if we don’t remove the treatment mean, the correlation we obtained does not reflect the true gene correlation.


\newpage
\section*{Poisson regression}
For a gene, let $Y= (Y_1, Y_2, \ldots, Y_n)$ be the gene expression level, and $X= (1, \ldots, 1, 0, \ldots, 0)$ be the indicator of whether sample is from treatment or control group. A Poisson regression model 
\begin{align*}\label{poisson}
Y_i\sim Pois(\mu_i) \\
\log(\mu_i) = \beta_0 +\beta_1 x_i
\end{align*}
The likelihood function 
\[L = \prod_{i=1}^n \frac{\mu_i^{y_i}}{y_i!}e^{-\mu_i}\]
And the log-likelihood function
\begin{align}\label{poissonLikelihood}
l(\beta_0, \beta_1) & = \log L = \sum_{i=1}^n(y_i\log \mu_i - \log y_i! -\mu_i) \\ \notag
 & = \sum y_i(\beta_0 +\beta_1x_i) - \sum \log y_i! - \sum \exp(\beta_0 + \beta_1x_i) \notag
\end{align}

\subsection*{Wald test}
The first derivative of (\ref{poissonLikelihood}) with respect to $\beta_0, \beta_1$
\[ \frac{\partial l}{\partial \beta_1} = \sum y_ix_i - \sum \exp(\beta_0 +\beta_1x_i)x_i\]
\[\frac{\partial l}{\partial \beta_0} = \sum y_i - \sum \exp(\beta_0 +\beta_1x_i)\]
The fisher information 
\[
I(\beta_0, \beta_1) = -E
\left[
\begin{array}{cc}
\frac{\partial^2 l }{\partial\beta_0^2} & \frac{\partial^2 l }{\partial\beta_0\partial \beta_1} \\
\frac{\partial^2 l }{\partial\beta_1\partial\beta_0} &  \frac{\partial^2 l }{\partial\beta_1^2} \\
\end{array}\right]
\]
Then the observed fisher information 
\[
\hat{I}(\beta_0, \beta_1) = 
\left[
\begin{array}{cc}
\sum \exp (\hat{\beta}_0 + \hat{\beta}_1 x_i)& \sum \exp (\hat{\beta}_0 + \hat{\beta}_1 x_i)x_i\\
\sum \exp (\hat{\beta}_0 + \hat{\beta}_1 x_i)x_i & \sum \exp (\hat{\beta}_0 + \hat{\beta}_1 x_i)x_i^2  \\
\end{array}\right]
\]
The Wald statistics for $H_0: \beta_1=0$ is therefore 
\[T = \frac{\partial l}{\partial \bm \beta} ^T [\hat{I}(\beta_0, \beta_1)]^{-1}  \frac{\partial l}{\partial \bm \beta}\]
Since there is no analytical solution for $\beta_1$, we don't know what the relation of $T$ and $Y$ is.

\subsection*{Score test} 

Testing $H_0:\beta_1=0$, the score test statistics is 
\[U = [Z(\tilde{\beta})^T I^{-1}(\tilde{\beta}) Z(\tilde{\beta})]^{1/2} \]
In this case, $\tilde{\beta} = (\beta_0, 0)$.   From (\ref{poissonLikelihood}) we have
\[\frac{\partial l}{\partial \beta_0} = \sum_i y_i - \sum_i \exp(\beta_0) \Rightarrow ~~ \hat{\beta}_0 = \log (\bar{y})\] 
Therefore
\[Z(\tilde{\beta}) = \left[
\begin{array}{c}
\sum_i y_i - \sum_i\exp(\beta_0 + \beta_1x_i) \\
\sum_i y_ix_i -\sum_i \exp(\beta_0 + \beta_1x_i)x_i
\end{array}
\right]|_{\beta_1 = 0}
= \left[
\begin{array}{c}
\sum y_i - \exp(\hat{\beta}_0) \\
\sum y_ix_i - \sum\exp(\hat{\beta}_0)x_i 
\end{array}
\right]
= \left[
\begin{array}{c}
0\\
\sum y_ix_i - \bar{y}\sum x_i 
\end{array}
\right]
\]
\[I(\tilde{\beta}) = 
\left[
\begin{array}{cc}
\sum \exp (\hat{\beta}_0 + \hat{\beta}_1 x_i)& \sum \exp (\hat{\beta}_0 + \hat{\beta}_1 x_i)x_i\\
\sum \exp (\hat{\beta}_0 + \hat{\beta}_1 x_i)x_i & \sum \exp (\hat{\beta}_0 + \hat{\beta}_1 x_i)x_i^2  \\
\end{array}\right]
=\left[\begin{array}{cc}
\sum y_i  & \bar{y}\sum x_i \\
\bar{y}\sum x_i  &\bar{y}\sum x_i^2\\
\end{array}\right]
\]
and it follows that 
\begin{align}\label{scoretest1}
U =  [Z(\tilde{\beta})^T I^{-1}(\tilde{\beta}) Z(\tilde{\beta})]^{1/2} =\left( \frac{n(\sum y_ix_i -\bar{y}\sum x_i)^2}{\bar{y}[n\sum x_i^2 -(\sum x_i)^2]}\right)^{1/2}
\end{align}
To make it simpler, let's assume the first $n/2$ elements of $\bm X$ are 1 and $\sum_{i=1}^{n/2}y_i \geq\sum_{i=n/2+1}^n y_i$, therefore we have $\sum x_i =\sum x_i^2=n/2$
\begin{align}\label{scoretest2}
U = \sqrt{\frac{n(\sum_{i=1}^{n/2}y_i-\bar{y}\cdot n/2)^2}{\bar{y}[n\cdot n/2 - (n/2)^2]}} = \sqrt{\frac{\frac{n}{2} (\bar{y}_1-\bar{y}_2)^2}{\bar{y}_1 + \bar{y}_2}} = \frac{\sqrt{\frac{n}{2}}(\bar{y}_1- \bar{y}_2)}{\sqrt{\bar{y}_1 + \bar{y}_2}}
\end{align}
where $\bar{y}_1 = \frac{\sum_{i=1}^{n/2}y_i }{n/2}$ and $\bar{y}_2 = \frac{\sum_{i=n/2 +1}^{n}y_i}{n/2}$ are just group means. This resembles a $t$ test statistics.

\subsection*{Simulation}
Generate correlated Poisson random variables. Let $X_0\sim Pois(\lambda_0), X_1 \sim Pois(\lambda_1), X_2\sim Pois(\lambda_2)$ and $X_0, X_1, X_2$ are mutually independent of each other. Let $Y_1= X_1 + X_0$, $Y_2= X_2 + X_0$, then $Y_1$ and $Y_2$ are correlated. 
\begin{align*}
Cov(Y_1, Y_2) &= Cov(X_1+X_0, X_2 + X_0) = Var(X_0) = \lambda_0
\end{align*}
The correlation between $Y_1$ and $Y_2$ can then be expressed by 
\[\rho(Y_1, Y_2) = \frac{Cov(Y_1, Y_2)}{\sqrt{Var(Y_1) Var(Y_2)}} = \frac{\lambda_0}{\sqrt{(\lambda_1 + \lambda_0)(\lambda_2 + \lambda_0)}}\]
Particularly, if we let $\lambda_1 = \lambda_2$, then $\rho = \frac{\lambda_0}{\lambda_0 + \lambda_1}$. More generally, $X_1\sim Pois((1-\rho) \lambda), X_0\sim Pois(\rho\lambda), X_2\sim Pois((1-\rho)\lambda)$,  then $Var(Y_1) = Var(Y_2) = \lambda$.  \\

\textbf{Note:} there is an upper bound for this correlation
\[\rho_{Y_1,Y_2}= \frac{Cov(Y_1, Y_2)}{\sigma_{Y_1}\sigma_{Y_2}} = \frac{\sigma_X^2}{\sigma_{Y_1}\sigma_{Y_2}} \leq \frac{\min (\sigma^2_{Y_1}, \sigma^2_{Y_2}) }{\sigma_{Y_1}\sigma_{Y_2}} = \min \left(
\frac{\sigma_{Y_1}}{\sigma_{Y_2}},\frac{\sigma_{Y_2}}{\sigma_{Y_1}} \right)\]
For each pair of $(\rho, \lambda)$, evaluate the sample correlation and score test statistics correlation UNDER THE NULL. 


\section*{Negative Binomial regression}

\subsection*{Simulation}
Generate correlated negative binomial random numbers. \\

\textbf{Lemma}   Let $X_0, X_1, X_2$ are i.i.d. negative binomial random variables, $X_i \sim NB(r_i, p)$ for $i=0, 1, 2$. Then $Y = \sum_i X_i \sim NB(\sum r_i,  p)$.  \\

This parameterization yields if $X\sim NB(r, p)$
\[E[X] = \frac{r(1-p)}{p} \]
\[Var[X] = \frac{r(1-p)}{p^2}\]
equation of mean-dispersion parameterization gives 
\[\frac{r(1-p)}{p} = \mu,  ~~~ \frac{r(1-p)}{p^2} = \mu + k\mu^2\]
where $\mu$ and $k$ are means and dispersion. Therefore
\begin{align} \label{NBparam}
p = \frac{1}{1 + k\mu}, ~~~ r = \frac{1}{k}
\end{align}

Let $Y_1 = X_0 + X_1$ and $Y_2 = X_0 + X_2$, then $Y_1 \sim NB(r_0 + r_1, p),   Y_2 \sim NB(r_0 + r_2, p)$.  We have 
\[Cov(Y_1, Y_2) = Var(X_0) = \frac{r_0(1-p)}{p^2}\]
and the correlation of $Y_1$ and $Y_2$
\[\rho_{Y_1Y_2} = \frac{Cov(Y_1, Y_2)}{\sqrt{Var(Y_1)Var(Y_2)}} = \frac{r_0}{\sqrt{(r_0 + r_1)(r_0 + r_2)}}\]

Particularly, if we let $r_1= r_2$, then $\rho = \frac{r_0}{r_1 + r_0}$.    \\

In simulation study, we let $X_0 \sim NB(\rho r, p)$, $X_1 \sim NB((1-\rho)r, p)$ and $X_2 \sim NB((1-\rho)r, p)$
and it follows that $Y_1 =X_0 + X_1 \sim NB(r, p)$, $Y_2 =X_0 + X_2 \sim NB(r, p)$, and $Cor(Y_1, Y_2) = \rho$.\\

For the null case, where there is no difference between treatment expression value $Y_{11}, \cdots, Y_{1,n/2}$ and control expression level $Y_{1, (n/2 +1)}, \cdots, Y_{1n}$ . From (\ref{NBparam}) we can see that $\mu$ should be a constant. Since my derivation of test statistic is under the assumption that dispersion is a constant for treatment/control, we also require $k$ thus $r$ to be constant.  (If we want to simulate DE cases, simply put $\mu_1\neq \mu_2$ or $p_1 \neq p_2$.) \\

For two genes, because of the way we simulate correlated NB data, it is required that $p$ remain constant within treatment (or control), which means the term $k\mu$ is constant. Note since $Y_1, Y_2$ are identically distributed (i.e. $r$ is the same for two genes), both $k$ and $\mu$ are the same for two genes. \\

To sum it up, we need $k$, $(\mu_1, \mu_2)$ and the desired correlation $\rho$ to the read count matrix under NB assumption.


\section*{ Sample correlation and Two sample T-test correlation}
\textbf{Conclusion}:  the correlation of $t$ test statistics converges to sample correlation if genes are NOT DE, and will generally be smaller than sample correlation if some genes are DE. \\

Under normal assumption. Let's assume gene 1 has expression level $G_1 = (X_{11}, \ldots, X_{1n}, Y_{11}, \ldots, Y_{1n})$ containing two treatments, similar for gene 2 $G_2 = (X_{21}, \ldots, X_{2n}, Y_{21}, \ldots, Y_{2n})$. Suppose in a general sense
\begin{equation}\label{assumption1}
  X_{i1}\sim N(\mu_1, \sigma^2_1), ~~ Y_{j1}\sim N(\mu_1 + \Delta_1, \sigma^2_1), ~~~X_{2i}\sim N(\mu_2, \sigma^2_2), ~~ Y_{2j}\sim N(\mu_2 + \Delta_2, \sigma^2_2)
\end{equation}
\subsection*{Gene-Gene correlation}
For the same gene, the samples are independent of each other. The correlation between two genes is defined, in my simulation, as 
\begin{align}\label{correlation}
\rho & = Cor(X_{1i}, X_{2i}) =  Cor(Y_{1j}, Y_{2j}) \\ \notag
\end{align}
Then $U_1= \bar{X}_1-\bar{Y}_1\sim N(\Delta_1, \frac{2\sigma_1^2}{n}),  U_2 = \bar{X}_2-\bar{Y}_2\sim N(\Delta_2, \frac{2\sigma_2^2}{n})$ 
It follows that 
\[ \bm U =\left( \begin{array}{c}
U_1\\
U_2\\
\end{array}\right)
\sim N\left[
\left(\begin{array}{c}
\Delta_1\\
\Delta_2\\
\end{array} \right), 
\frac{2}{n}\left(
\begin{array}{cc}
\sigma_1^2 &\rho \sigma_1\sigma_2 \\
\rho \sigma_1 \sigma_2 & 	\sigma_2^2 \\
\end{array}
\right)
 \right]
\]
and define
\[S_{X_1}^2 = \frac{1}{n-1}\sum_{i=1}^n(x_{i1}-\bar{x}_1)^2 \Rightarrow \frac{(n-1)S_{X_1}^2}{\sigma_1^2} \sim \chi^2(n-1)\]
\[S_{Y_1}^2 = \frac{1}{n-1}\sum_{i=1}^n(y_{i1}-\bar{y}_1)^2 \Rightarrow \frac{(n-1)S_{Y_1}^2}{\sigma_1^2} \sim \chi^2(n-1)\]
\[S_1^2 = \frac{(n-1)S_{X_1}^2 + (n-1)S_{Y_1}^2}{n-1 + n-1} = \frac{S_{X_1}^2 + S_{Y_1}^2}{2}\]
\[
\bm W_X =\left( \begin{array}{c}
W_{X_1}\\
W_{X_2}\\
\end{array}\right)
=
\left( \begin{array}{c}
(n-1)S_{X_1}^2/\sigma_1^2\\
(n-1)S_{X_2}^2/\sigma^2_2\\
\end{array}\right)
\sim \text{bivariate } \chi^2 \text{ distribution}
\]
therefore  the distribution of $W = \frac{W_X + W_Y}{2}$ can be derived from bivariate $\chi^2$ distribution, and it's not related to the mean parameter $(\mu_1, \mu_2, \Delta_1, \Delta_2)$. By Basu's theorem, $\bm U$ is complete sufficient for mean, and $W$ is ancillary for mean, then $\bm U$ is independent of $\bm W$. Subsequently $(U_1, U_2)$ is independent of $(S_1, S_2)$. 




\end{document}

